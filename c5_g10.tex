\section{Game No.~10}
\begin{center}
Ruy Lopez \\
\end{center} 
\begin{multicols}{2}
\noindent White \hfill Black \\
\noindent Duras \hfill Dr Bernstein

\newgame

\noindent\mainline[style=styleC,level=1]{1. e4 e5 2. Nf3 Nc6 3. Bb5 Nf6 4. d3 d6 5. c4}

\begin{center}
\vspace{-0.5cm}
\chessboard[smallboard,showmover=false]
\vspace{-0.1cm}
\end{center} 

\noindent
A similar line of play to this was adopted by Anderssen against Steinitz, but refuted by the latter. It is clear that the point d4 becomes weak.  

\mainline{5... g6 6. d4 exd4 7. Nxd4 Bd7 8. Nc3 Bg7 9. Bxc6 bxc6 10. Bg5 h6 11. Bh4 O-O 12. O-O Re8 13. Re1 Rb8 14. Rb1 c5 15. Nb3}

\noindent
A surprisingly weak move. The knight is out of play here. On f3, he would have been of better use, as e4--e5 was first of all threatened. At all events, \variationNM[invar]{Nf3} would have prevented Black's \variationNM{Bc6}, for after \variation{15. Nf3 Bc6 16. e5 Bxf3 17. Qxf3 dxe5 18. Rbd1} Black would be lost. \variation{18... Qe7 19. Nd5 }

\mainline[outvar]{15... Bc6}

\noindent
Prevents \variationNM[invar]{Nd5} because of \variationNM{...g5}, gaining the e-pawn. 

\begin{center}
\vspace{-0.5cm}
\chessboard[smallboard,showmover=false]
\vspace{-0.1cm}
\end{center} 

\mainline[outvar]{16. Qd3 Qc8}

\noindent
The commencement of an attack conducted equally well from a strategical and tactical point of view. 

\mainline{17. Nd2 Nd7 18. b3 Qa6 19. Qc2 Qa5 20. Ne2}

\noindent
If \variation[invar]{20. Nd5 Bxd5 21. cxd5 Qc3 22. Rbc1 Qxc2 23. Rxc2 g5 24. Bg3 f5 25. f3 f4 26. Bf2 Ne5} and Black's game would, at least, not have been inferior. After the text, however, White appears to be irretrievably lost. 

\mainline[outvar]{20... Nf8 21. f3 Ne6 22. Bf2 }

\begin{center}
\vspace{-0.5cm}
\chessboard[smallboard,showmover=false]
\vspace{-0.1cm}
\end{center} 

\mainline{22... Bd7}

\noindent
Intending to play \variationNM[invar]{Nd4}. But first he renders the c-pawn mobile. 

\mainline[outvar]{23. Nf1 Nd4 24. Qd3 Nc6 25. Nc1 Qa3}

\begin{center}
\vspace{-0.5cm}
\chessboard[smallboard,showmover=false]
\vspace{-0.1cm}
\end{center} 

\noindent
Brilliant play. The a-pawn is thus fixed in its weak position. 


\mainline{26. Ne3 Nb4 27. Qd2 a5 28. Nd5 Nxd5 29. exd5}

\noindent
This loses forthwith. If he had retaken \variation[invar]{29. cxd5} Black would have continued \variation{29... c4} threatening to establish a most dangerous passed pawn at c3. \variation{30. Bd4} Would then have been a mistake, as after \variation{30... Qb4 31. Rd1} (\variation[invar]{31. Qxb4? Bxd4+})
\variation[outvar]{31... c3} Black would have won at once. Black's play in this game is of the highest order.  

\begin{center}
\vspace{-0.5cm}
\chessboard[smallboard,showmover=false]
\vspace{-0.1cm}
\end{center} 

\vfill\null
\columnbreak

\mainline[outvar]{29... Rxe1+ 30. Bxe1 Bf5 31. Nd3 Bxd3 32. Qxd3 Qxa2 33. h3 a4 34. b4 cxb4 35. Rxb4 Rxb4 36. Bxb4 Qb3 37. Qd2 a3 38. Bxa3 Qxa3}

White resigns\\

\begin{center}
\vspace{-0.75cm}
\noindent 1h 37 \hspace{2cm} 1h 10 \\
\vspace{-.25cm}\noindent\rule{3cm}{0.4pt}
\end{center}

\vfill\null


\end{multicols}