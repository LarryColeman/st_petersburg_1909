\section{Game No.~6}
\begin{center}
Ruy Lopez \\
\end{center} 
\begin{multicols}{2}
\noindent White \hfill Black \\
\noindent Forg\'acs \hfill Speijer

\newgame

\noindent\mainline[style=styleC,level=1]{1. e4 e5 2. Nf3 Nc6 3. Bb5 Nf6 4. O-O Be7 5. Nc3 d6 6. d4 exd4 7. Bxc6+ bxc6 8. Nxd4 Bd7 9. b3}

\noindent
This development is a little too slow, as Mr.~Speijer proves.

\begin{center}
\vspace{-0.5cm}
\chessboard[smallboard,showmover=false]
\vspace{-0.1cm}
\end{center} 

 

\mainline{9... O-O 10. Bb2 Re8 11. Qf3}

\noindent
Not a good conception. 

\mainline{11... Bf8 12. h3 g6}

\noindent
Thus white's dark squared bishop is counterbalanced by Black's, while, at the same time, the pawn at g6 prevents the entry of the knight at f5. 

\mainline{13. Nde2 Bg7 14. Ng3}

\noindent
Black was threatening \variationNM[invar]{Nxe4}. 

\begin{center}
\vspace{-0.5cm}
\chessboard[smallboard,showmover=false]
\vspace{-0.1cm}
\end{center}

\mainline[outvar]{14... h5}

\noindent
Fine and energetic play. 

\mainline{15. Rfe1 Nh7 16. Na4 Ng5 17. Qd3 Bxb2 18. Nxb2 Qf6 19. c3 Rad8}

\noindent
Black could here give the game a turn in his favour by \variation[invar]{19... Bxh3 20. gxh3 Nxh3+ 21. Kh2 Qh4 22. Qf1} 
(\variation[invar]{22. Qe3 Nf4+ 23. Kg1 Qg4})
\variation[outvar]{22... Nxf2+ 23. Kg2 Ng4 24. Qh1 Qg5} With a double threat of \variationNM{Qd2+} and h5-h4. 

\mainline[outvar]{20. Nc4 h4}

\noindent\variationNM[invar]{Bxh3} would still have been strong, for Black would rather easily get four pawns for the piece with a good position. 

\mainline[outvar]{21. Nf1 Qf4 22. Qd2 Qxd2 23. Ncxd2 Ne6 24. Nf3 g5 25. Ne3 f6 26. Ng4 Kg7 27. Nd4 Kg6 28. f3 Ng7 }

\begin{center}
\vspace{-0.5cm}
\chessboard[smallboard,showmover=false]
\vspace{-0.1cm}
\end{center}

\mainline{29. Ne3 f5 30. exf5+ Kf7 31. b4 c5 32. bxc5 dxc5 33. Nb3 Nxf5 34. Ng4}

\begin{center}
\vspace{-0.5cm}
\chessboard[smallboard,showmover=false]
\vspace{-0.1cm}
\end{center}

\noindent
After \variation[invar]{34. Nxf5 Bxf5 35. Nxc5 Rxe1+ 36. Rxe1 Rd2 37. a4 Rc2 38. Re3 Kf6} White cannot win, as his king cannot come into play. 

\mainline[outvar]{34... c4 35. Nc5 Bc8 36. Ne5+ Kf6 37. Nxc4 Nd6 38. Rxe8 Nxe8 39. Kf2 Ng7 40. Rb1 Bf5 41. Rb7 Ne6 42. Nxe6}

\noindent
It would have given better chances to keep the minor pieces: \variation[invar]{42. Nb3 Rd3 43. Ne3} to White's advantage. 

\begin{center}
\vspace{-0.5cm}
\chessboard[smallboard,showmover=false]
\vspace{-0.1cm}
\end{center}

\mainline[outvar]{42... Bxe6 43. Rxc7 Bxc4 44. Rxc4 Rd2+ 45. Ke3 Rxa2 46. Rg4 a5 47. f4 gxf4+ 48. Kf3}

\begin{center}
\vspace{-0.5cm}
\chessboard[smallboard,showmover=false]
\vspace{-0.1cm}
\end{center}

%%%%%
\vfill\null
\columnbreak
%%%%%

\noindent
\variation[invar]{48. Kxf4} would have led to nothing, e.g. \variation{48... Ra4+ 49. Ke3 Rxg4 50. hxg4 Kg5 51. c4 Kxg4 }

\mainline[outvar]{48... Rc2 49. Rxf4+ Kg5 50. Rg4+ Kh5 51. Rc4 a4 52. Rxa4 Rxc3+ 53. Kf4 Rg3 54. Ra8 Kh6}

\begin{center}
\vspace{-0.5cm}
\chessboard[smallboard,showmover=false]
\vspace{-0.1cm}
\end{center}

\begin{center}
Drawn\\
\end{center} 
%No game time info in the book
%\noindent 4h 12 \hfill 4h 24 \\
\begin{center}
\vspace{-.5cm}\noindent\rule{3cm}{0.4pt}
\end{center}

\vfill\null


\end{multicols}