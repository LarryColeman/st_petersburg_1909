\section{Game No.~4}
\begin{center}
Ruy Lopez \\
\end{center} 
\begin{multicols}{2}
\noindent White \hfill Black \\
\noindent Teichmann \hfill Vidmar

\newgame

\noindent\mainline[style=styleC,level=1]{1. e4 e5 2. Nf3 Nc6 3. Bb5 Nf6 4. O-O d6 5. d4 Bd7 6. Nc3 Be7 7. Re1 exd4 8. Nxd4 O-O 9. Nde2}

\begin{center}
\vspace{-0.7cm}
\chessboard[smallboard,showmover=false]
\vspace{-0.1cm}
\end{center} 

\noindent\vspace{-0.1cm}
\variation[invar]{9. Bg5} looks the natural move. 

\mainline[outvar]{9... Re8 10. Ng3 Bf8 11. b3 g6 12. Bb2 Bg7 13. Nd5 a6 14. Bxc6}

\noindent
To \variation[invar]{14. Bf1} Black would have replied \variation{14... Ne5} and if \variation{15. f4?} he would have played \variation{15... Nxd5} follwed by \variation{16...Nf3+} or \variation{16...Nf4} according to circumstances, with a good game. 

\vspace{-0.3cm}\mainline[outvar]{14... Bxc6 15. Nxf6+ Bxf6 16. Bxf6 Qxf6 17. Qd3 Re6 18. f3 Rae8 19. c4 Qe7}

\vspace{-0.25cm}\noindent
Threatening \variation[invar]{20...f5}; White would probably reply \variation{20. Rad1}. After that it appears for both players an almost hopeless undertaking to drive the opponent from his position. 

\begin{center}
\vspace{-0.2cm}
\small
(The final position.)\\
\vspace{-0.5cm}
\chessboard[smallboard,showmover=false]
\vspace{-0.1cm}
\end{center} 

\begin{center}
\vspace{-0.25cm}
Drawn\\
\end{center} 
\vspace{-0.25cm}
\noindent 1h 07 \hfill 1h 00 \\
\begin{center}
\vspace{-.8cm}\noindent\rule{3cm}{0.4pt}
\end{center}

\end{multicols}