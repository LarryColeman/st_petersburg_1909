\section{Game No.~8}
\begin{center}
Queen's Gambit Declined \\
\end{center} 
\begin{multicols}{2}
\noindent White \hfill Black \\
\noindent von Freymann \hfill Tartakower

\newgame

\noindent\mainline[style=styleC,level=1]{1. d4 c5}

\noindent
After this White does not seem to have anything better than to turn into the Sicilian Defence by \variation[invar]{1... c5 2. e4}. After \variation{2... cxd4 3. Nf3 e5? 4. c3} White gets sufficient compensation for the pawn sacrificed. 

\noindent
\variation{1... c5 2. d5} deserves consideration, as the pawn is here in a secure position, and White succeeds in hampering Black's game a little, without having lost time. 

\mainline[outvar]{2. e3 d5 3. c4 e6 4. Nf3 Nf6 5. Nc3 a6 6. cxd5 Nxd5 7. Bd3 Nc6 8. O-O cxd4 9. exd4 Be7 10. Re1 O-O 11. Be3 b5}

\begin{center}
\vspace{-0.5cm}
\chessboard[smallboard,showmover=false]
\vspace{-0.1cm}
\end{center} 

\noindent
A venturesome move. He risks \variation[invar]{11... b5 12. Qc2 Ncb4 13. Bxh7+ Kh8 14. Qb1 g6 15. Bxg6 fxg6 16. Qxg6} whereupon White would have already three pawns for the piece with a good attack. 

\mainline[outvar]{12. Rc1 Bb7 13. Ne4 Nxe3 14. fxe3 Nb4}

\begin{center}
\vspace{-0.5cm}
\chessboard[smallboard,showmover=false]
\vspace{-0.1cm}
\end{center}

\mainline{15. Nc5}

\noindent
After \variation[invar]{15. Nc3} White would have quite a good position. 

\mainline[outvar]{15... Bxf3 16. gxf3 Nxa2 17. Ra1 Nb4 18. Be4 Ra7 19. f4 Qb6}

\noindent
The logical winning continuation was \variation[invar]{19... Bxc5 20. dxc5 Qxd1 21. Raxd1 f5}. White's c-pawn could not be held, whilst Black would defend his e-pawn comfortably with the king. 

\mainline[outvar]{20. Nd3 Nd5 21. Bxd5 exd5 22. Re2 Re8 23. Rg2 Qe6 24. Ne5 f6}

\noindent
This move required exact calculation. It was necessary to dislodge the knight, or else White would have played \variationNM[invar]{Qf3} and f4-f5. 

\mainline[outvar]{25. Qh5 Bf8 26. f5 Qe7 27. Ng4 Rc8 }

\begin{center}
\vspace{-0.5cm}
\chessboard[smallboard,showmover=false]
\vspace{-0.1cm}
\end{center}

\mainline{28. Rg3}

\noindent
If \variation[invar]{28. Rxa6} Black would not, by any means, reply \variation{28...Rxa6}, on account of \variation{29. Nh6+} and \variation{30. Nf7+} giving perpetual check, but \variation{28... Rc1+ 29. Kf2 Rc2+ 30. Kf1 Rxa6 31. Rxc2} (Necessary to cover the mate) \variation{31... Qe4 }.


\mainline[outvar]{28... Kh8 29. Nf2 Rac7 30. Rxa6 Rc2 31. Re6 Qb4 32. Rf3 Rxb2 33. Kg2 Rbc2 34. Rh3 h6 35. Rg3 Qe1 36. Rf3 b4 37. Rb6 b3}

\noindent
A pretty combination, which decides the game at once. 

\mainline{38. Rxb3 Qd1 39. Rb6 Rxf2+ 40. Kxf2 Rc2+ 41. Kg3 Qg1+ 42. Kf4 Rxh2 43. Qg4}

\noindent
Or \variation[invar]{43. Rh3 Qf1+ 44. Rf3 Qh1 }

\mainline[outvar]{43... Qh1 44. Rb8 Kg8 45. e4 Rh4 46. e5 h5}

\noindent
White resigns.\\
\begin{center}
\vspace{-0.5cm}
\noindent 2h 50 \hspace{2cm} 1h 50 \\
\vspace{-.25cm}\noindent\rule{3cm}{0.4pt}
\end{center}

\vfill\null


\end{multicols}