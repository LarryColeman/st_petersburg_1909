\section{Game No.~3}
\begin{center}
Queen's Gambit Declined \\
\end{center} 
\begin{multicols}{2}
\noindent White \hfill Black \\
\noindent Nenarokow \hfill Dr.~Perlis

\newgame

\noindent\mainline[style=styleC,level=1]{1. d4 d5 2. c4 e6 3. Nc3 Nf6 4. Nf3 Be7 5. Bf4 O-O 6. e3 c5 7. Bd3 Nc6 8. cxd5 exd5 9. dxc5 Bxc5 10. O-O Be6 11. Rc1 Rc8}

\begin{center}
\vspace{-0.5cm}
\chessboard[smallboard,showmover=false]
\vspace{-0.1cm}
\end{center} 

\noindent 
Better \variation[invar]{11... a6 12. Bb1 d4 13. Na4 Ba7} ; the black dark-squared bishop should exert a pressure on d4. 

\mainline[outvar]{12. Bb1 Na5}

\noindent
There the knight is out of play. \variation[invar]{12... Qe7 13. Bg5 Rfd8 14. Qd3 h6} was a feasible line of play. The checks would have done Black no harm. 

\mainline[outvar]{13. Bg5 Be7 14. Nd4 g6}

\begin{center}
\vspace{-0.5cm}
\chessboard[smallboard,showmover=false]
\vspace{-0.1cm}
\end{center} 

\mainline{15. Qe2}

\noindent 
White might have played f4 followed by f5; e.g. \variation[invar]{15. f4 Bg4 16. Qe1 Nc4 17. f5 Nxb2 18. h3} and White would have an irresistible attack. 

\begin{center}
\vspace{-0.5cm}
\chessboard[smallboard,showmover=false]
\vspace{-0.1cm}
\end{center}

\mainline[outvar]{15... Nc6 16. Nf3 Qb6 17. h3 Rfd8 18. Rfd1 Kg7 19. Nd4 Nxd4 20. exd4 Rc4 21. Be3 Rdc8 22. Bd3 Rb4 23. b3 Qd8 24. Na4 Rxc1 25. Rxc1 Bd7 26. Nc5 Rb6 27. Bf4 Bxc5 28. dxc5 Re6 29. Qb2 Qe7 30. Bd6 Qe8 31. Qd2 Bc6 32. Bf4 Ng8 33. Qc3+ f6 34. Kh2 Kf7 35. Qd2 a6 36. Bd6 Kg7 37. Bf4 Qe7}

\noindent
Adjourned. 

\begin{center}
\vspace{-0.5cm}
\chessboard[smallboard,showmover=false]
\vspace{-0.1cm}
\end{center}

\mainline{38. Bd6 Qe8 39. Bf4 Qe7 40. b4 Qe8 41. a3 Kf7 42. Rb1 f5 43. Rb2 Nf6 44. Bb1 Qe7 45. f3 Nh5 46. Bd6 Qh4 47. g3}

\noindent
Both parties have taken care not to alter the position to any considerable extent. Black here lays a trap. If \variation[invar]{47. Qh6} Black would have answered \variation{47... Rxd6}

\mainline[outvar]{47... Qd8 48. Ba2 Nf6 49. Kg2 Qe8 50. Kf2 Kg7 51. Bf4 Bb5}

\noindent
An altogether faulty manoeuvre; the attack thus imitated is easily parried, whilst the d-pawn is left without support. 

\mainline{52. Bh6+ Kh8 53. Qd1 Ng8 54. Qd4+ Nf6 55. h4}

\noindent
This was calculated to a nicety. 

\begin{center}
\vspace{-0.5cm}
\chessboard[smallboard,showmover=false]
\vspace{-0.1cm}
\end{center}

%%%%%%%%%%%%%%%%%%%
\vfill\null
\columnbreak
%%%%%%%%%%%%%%%%%%%

\mainline{55... Re2+ 56. Kg1 Re1+ 57. Kh2 Re2+ 58. Kh3 Qe6 59. Bg5 f4+ 60. g4 Re5}

\noindent
Black here lost the game by exceeding the time limit. The game might have gone on as follows: \variation[invar]{60... Re5 61. Qxf4 Bf1+ 62. Kh2 Nd7 63. Qd4} to White's advantage. 

\begin{center}
\vspace{-0.5cm}\noindent\rule{3cm}{0.4pt}
\end{center}

\end{multicols}