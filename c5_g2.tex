%\vspace{0.5cm}
\section{Game No.~2}
\begin{center}
Vienna Game \\
\end{center} 
\begin{multicols}{2}
\noindent White \hfill Black \\
\noindent E. Cohn \hfill Burn

\newgame

\noindent\mainline[style=styleC,level=1]{1. e4 e5 2. Nc3 Bc5 3. g3 Nf6 4. Bg2 d6}

\begin{center}
\vspace*{-1cm}
\chessboard[smallboard,showmover=false]
\vspace{-0.1cm}
\end{center}

\noindent
\variation[invar]{4... Nc6} appears to be preferable, with a view to saving the important dark-squared bishop from being exchanged, by \variation{5...a6 }

\mainline[outvar]{5. Na4 Nc6 6. Ne2 Qe7 7. d3 Be6 8. O-O d5 9. Nxc5 Qxc5 10. Be3 Qd6 11. exd5 Bxd5 12. Nc3 Bxg2 13. Kxg2 Nd5 14. Qd2}

\noindent
\variation[invar]{14. Qf3} taking posession of the diagonal which the fianchettoed bishop commanded before, seems more natural.

\mainline[outvar]{14... O-O}

\noindent
Black ought to castle queenside, to attack on the kingside. 

\mainline{15. Ne4}

\begin{center}
\vspace{-0.5cm}
\chessboard[smallboard,showmover=false]
\vspace{-0.1cm}
\end{center}

\mainline{15... Nxe3+}

\noindent
\variation[invar]{15... Qg6 16. f4 f5 17. Nc3} (or \variation{17. Nc5 Nxe3+ 18. Qxe3 Nd4
17... Rad8}) would have created interesting complications, which would probably have turned out in Black's favour. 

\mainline[outvar]{16. Qxe3 Qd4 17. c3 Qxe3 18. fxe3}

\begin{center}
Drawn\\
\end{center} 
\begin{center}
\vspace{-.5cm}
\noindent 1h 10 \hspace{2cm} 0h 40 \\
\vspace{-.25cm}\noindent\rule{3cm}{0.4pt}
\end{center}

\end{multicols}