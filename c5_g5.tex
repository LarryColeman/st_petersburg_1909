\section{Game No.~5}
\begin{center}
Ruy Lopez \\
\end{center} 
\begin{multicols}{2}
\noindent White \hfill Black \\
\noindent Schlecter \hfill Dr.~Lasker

\newgame

\noindent\mainline[style=styleC,level=1]{1. e4 e5 2. Nf3 Nc6 3. Bb5 Nf6 4. O-O d6 5. d4 Bd7 6. Nc3 Be7 7. Re1 exd4 8. Nxd4 O-O 9. Nde2 a6 10. Ba4}

\begin{center}
\vspace{-0.5cm}
\chessboard[smallboard,showmover=false]
\vspace{-0.1cm}
\end{center} 

\noindent
The retreat to d3 appears stronger. It is true that Black can then exchange white's bishop for a knight, by playing, say, \variation[invar]{10...Ne5}; but in that case White would retake with the c-pawn and would have two strong pawns in the centre as compensation for Black's two bishops. 

\mainline[outvar]{10... Re8 11. f3 h6}

\noindent
If black played \variation[invar]{11... Bf8} at once, White's reply would be \variation{12. Bg5} threatening \variation{13. Nd5}. After this Black would have nothing better than \variation{12... h6 13. Bh4 Be7 }

\mainline[outvar]{12. Be3 Bf8 13. Qd2 Ne5}

\noindent
By this move, Black frees his game. 

\mainline{14. Bb3 }

\begin{center}
\vspace{-0.5cm}
\chessboard[smallboard,showmover=false]
\vspace{-0.1cm}
\end{center} 

\noindent
It was not good to retire the Bishop. White ought to have exchanged and developed his game further by \variationNM[invar]{Rad1}. 

\mainline[outvar]{14... c5 15. Bd5 Rb8 16. Nf4 b5 17. a3 Nxd5 18. Ncxd5 f5}

\noindent
The point of Black's strategy. After the exchange of the e-pawn, the weakness of the d-pawn does not signify. 

\mainline{19. exf5 Bxf5 20. Bf2 Qd7 21. Ne3 Bh7 22. Nfd5 Qf7 23. Rad1 Nc6 24. Bg3 Rbd8 25. Bh4 Rd7 26. Ng4 Rxe1+ 27. Rxe1 Nd4}

\noindent
Decisive. 

\begin{center}
\vspace{-0.5cm}
\chessboard[smallboard,showmover=false]
\vspace{-0.1cm}
\end{center}

\mainline{28. Nge3}

\noindent
White dare not reply \variation[invar]{28. Nde3} as \variation{28... g5 29. Bg3 h5} would get him into difficulties. 

\mainline[outvar]{28... Bxc2 29. Nxc2 Nxc2 30. Nf6+ gxf6 }

\begin{center}
\vspace{-0.5cm}
\chessboard[smallboard,showmover=false]
\vspace{-0.1cm}
\end{center}

\mainline{31. Qxc2}

\noindent
Intending to take up a strong position by Qf5. 

\mainline{31... f5 32. f4 Bg7}

\noindent\variation[invar]{32... d5 33. Re5 d4} and, whether queen or rook captures f5, ...d3 would have decided the game at once. The text is therefore loss of time.

\mainline[outvar]{33. h3 c4 34. g4}

\noindent
A desperate attempt to obtain an attack.  

\mainline{34... d5}

\noindent
Simply \variation[invar]{34... fxg4 35. hxg4} follwed either by \variationNM{Qxf5} or \variationNM{d5} was indicated. 

\mainline[outvar]{35. gxf5 d4 36. Qe4 d3 37. f6 Bf8}

\noindent
If \variation[invar]{37... Bxf6} ; \variation{38. Bxf6 Qxf6 39. Qe8+ }

\mainline[outvar]{38. Kh2 d2}

Better \variation[invar]{38... Kh8 39. Rg1 a5} in order to play ...b5 and ...c6, which was feasible in spite of f4-f5 and \variationNM{Qe6}. 

\begin{center}
\vspace{-0.5cm}
\chessboard[smallboard,showmover=false]
\vspace{-0.1cm}
\end{center}

\noindent
Adjourned.

\mainline[outvar]{39. Rd1 Qh5}

\variation[invar]{39... Kh8} Was still the right move. If white plays \variation{40. Qe2} then \variation{40... Bd6 41. Kg3 Qg6+ 42. Qg4 Bxf4+} and wins; likewise after \variation{39... Kh8 40. Qe2 Bd6 41. Qxd2 Bxf4+ 42. Qxf4 Rxd1 43. Qxh6+ Kg8 44. Qg5+ Kh7} White's checks would cease and Black should win. 

\mainline[outvar]{40. Qe6+ Kh8 41. f7}

\noindent
This clever move threatens Bf6+ 
 
\mainline{41... Qxf7 42. Bf6+}

\noindent
Far better than at once \variation[invar]{42. Qxf7} as White's b-pawn is saved from attack by the exchange of the Bishops.  

\mainline[outvar]{42... Bg7}

\noindent
To \variation[invar]{42... Kh7} the reply would not have been \variation{43. Qf5+ Kg8 44. Rg1+} as after \variation{44... Bg7 45. Rxg7+ Qxg7 46. Bxg7 d1=Q} Black would get out of the checks and win;  but after \variation{42... Kh7 43. Qxf7+ Rxf7 44. Bc3} White would have taken up a strong defensive position.

\begin{center}
\vspace{-0.5cm}
\chessboard[smallboard,showmover=false]
\vspace{-0.1cm}
\end{center}

\mainline[outvar]{43. Qxf7 Rxf7 44. Bxg7+ Kxg7 45. Rxd2 Rxf4}

\noindent
Black certainly remains with a pawn to the good, but White threatens to break up the pawns by a4. After an endgame, which is is played by White in a sensible manner and which needs no comment, the game now ends in a draw. 

\mainline{46. Kg3 Re4 47. Kf3 Re1 48. Rd7+ Kf6 49. Rd6+ Ke5 50. Rxa6 Rb1 51. Rxh6 Rxb2 52. Ke3 Rb3+ 53. Kd2 Kd4 54. Rd6+ Kc5 55. Ra6 Rxh3 56. Ra8 Rh2+ 57. Kc3 Ra2 58. Rc8+ Kb6 59. Rb8+ Kc6 60. Ra8 Kc5 61. Rc8+ Kb6 62. Rb8+ Kc6 63. Ra8 Kb7 64. Ra5 Kb6 65. Ra8 Kc6 66. Kd4 Kb7 67. Ra5 Rd2+ 68. Kc3 Rd5 69. Kb4 Kb6 70. a4 c3 71. axb5}

\noindent
If \variation[invar]{71. axb5 c2} then follows \variation{72. Ra6+} and \variation{73. Rc6}.

\begin{center}
Drawn\\
\end{center} 
\begin{center}
\vspace{-.5cm}
\noindent 4h 12 \hspace{2cm} 4h 24 \\
\vspace{-.25cm}
\noindent\rule{3cm}{0.4pt}
\end{center}

\end{multicols}