\section{Game No.~9}
\begin{center}
Ruy Lopez \\
\end{center} 
\begin{multicols}{2}
\noindent White \hfill Black \\
\noindent Spielmann \hfill Salwe

\newgame

\noindent\mainline[style=styleC,level=1]{1. e4 e5 2. Nf3 Nc6 3. Bb5 Nf6 4. O-O d6 5. d4 Bd7 6. Nc3 Be7 7. Re1 exd4 8. Nxd4 O-O 9. Bxc6}

\begin{center}
\vspace{-0.5cm}
\chessboard[smallboard,showmover=false]
\vspace{-0.1cm}
\end{center} 

\noindent
This exchange leads to nothing, except, perhaps, that it prevents Black from exchanging both knight and bishop. This, however, need not be feared. 

\mainline{9... bxc6 10. b3 Re8 11. Bb2 Bf8 12. Qd3 g6 13. Nde2}

\noindent
This strategical manoeuvre is altogether wrong. White might, at this juncture, play \variation[invar]{13. Rad1} and answer \variation{13... Bg7} with \variation{14. f4}. Though the pawns at e4 and f4 are then exposed to attacks, they are not weak, and assist in maintaining the balance of position. 

\mainline[outvar]{13... Bg7 }

\begin{center}
\vspace{-0.5cm}
\chessboard[smallboard,showmover=false]
\vspace{-0.1cm}
\end{center} 

\mainline{14. Ng3}

\noindent
Since black has already moved the pawn to g6, the knight is not favourably posted on this square. 

\mainline{14... h5}

\noindent
A splendid strategical idea. From this insignificant beginning, Black obtains strong pressure on the kingside. 

\mainline{15. Rad1 h4 16. Nf1 Nh5 17. Bc1 Be5 18. Ne2 g5 19. g3 Qf6 20. Qe3 g4 21. Nd2 d5}

\noindent
If Black had played \variation[invar]{21... Be6} here, White would have been at a loss for what to do. If, perchance, \variation{22. Rf1} to prepare f2-f4, Black replies \variation{22... Kh8} and the advance of the f-pawn would then only open the lines for Black's rooks and bishops. 

\noindent
If \variation{21... Be6 22. Qd3} then \variation{22... d5 23. Qa6? Bc8}. In any case, White would have been in a precarious position.
 
\begin{center}
\vspace{-0.5cm}
\chessboard[smallboard,showmover=false]
\vspace{-0.1cm}
\end{center} 

\mainline[outvar]{22. Nc4}

\noindent
By exchanging one of the two bishops, White frees his game and now forces the draw, with correct judgement of the situation. 

\mainline{22... hxg3 23. fxg3 Qg6 24. Nxe5 Rxe5 25. Nf4 Nxf4 26. Qxf4 Rae8 27. Bb2 Rxe4 28. Rxe4 Qxe4 29. Qg5+ Qg6 30. Qh4 Qh7 31. Qg5+ }

\begin{center}
\vspace{-0.5cm}
\chessboard[smallboard,showmover=false]
\vspace{-0.1cm}
\end{center}

\begin{center}
Drawn.\\
\end{center}
\begin{center}
\vspace{-0.5cm}
\noindent 1h 37 \hspace{2cm} 1h 10 \\
\vspace{-.25cm}\noindent\rule{3cm}{0.4pt}
\end{center}

\vfill\null


\end{multicols}